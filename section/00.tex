\section{Vorbemerkungen}
Die Nachfolgenden Erläuterungen beschreiben die Installation und Konfiguration einer Eclipse-Arbeitsumgebung für jBPM im Zusammenspiel mit einem Wildfly Application Sever und einer PostgreSQL-Datenbanksystem.

\subsection{Typographische Stilmittel}
Für eine bessere Verwendung dieses Dokumentes -- und eventuellen Folgedokumenten -- wurde auf eine konsistente Nutzung typographischer Stilmittel geachtet, welche im folgenden kurz erläutert werden sollen.

\subsubsection{Informationsboxen}
Anmerkungen und weiterführende Informationen zum aktuell besprochenen Thema -- welche nicht zwingend für die Bearbeitung des aktuellen Kapitels, Abschnitts bzw. Unterabschnitts notwendig sind -- werden jeweils am Ende in einer separaten Informationsbox eingefügt. Der Inhalt kann sich je nach Bedarf unterscheiden und wird über eine Variation verschiedener Symbole in der oberen, rechten Ecke der Informationsbox beschrieben. Besonders bei längeren Informationsboxen soll dies dem Leser helfen, den darin befindlichen Inhalt schnell zu überblicken und die freie Entscheidung über ein möglicherweise gewünschtes Lesen beschleunigen. Die unten stehende Informationsbox stellt das Aussehen beispielhaft dar und bietet eine Übersicht aller verwendeten Symbole sowie deren Bedeutung.
\begin{info}{111}
	Erläuterungen zur Bedeutung der Symbole:\\[1em]
	\begin{tabular}{m{2em}m{0.88\textwidth}}
	\bookmarkicon	& deutet an, dass die Informationsbox Verweise zu externen Quellen mit zusätzlichen Informationen zu diesem Thema beinhaltet. \\[0.5em] 
	\downloadicon	& deutet an, dass die Informationsbox einen oder mehrere Verweise zu externen Download-Quellen beinhaltet. \\[0.5em]
	\fileicon	& deutet an, dass die Informationsbox zusätzliche Erläuterungen zu dem im aktuellen Kapitel, Abschnitt oder Unterabschnitt beinhaltet.
	\end{tabular}
\end{info}
\subsubsection{Quellcode-Auszüge}
Quellcode-Beispiele werden unabhängig von der verwendete Programmier- bzw. Script-Sprache einheitlich dargestellt. Beispielhaft soll dies im unten zu sehenden Listing \ref{lst:listing-example} repräsentiert werden. 
\javalisting{HelloWorld.java}{listing-example}
Im Text wird auf Details im Listing über die Zeilennummerierung Bezug genommen. Die Nummerierung befindet sich am linken Rand und beginnt stets mit 1. Es sei an dieser Stelle ausdrücklich darauf hingewiesen, dass die im Listing zu sehende Zeilennummerierung ausschließlich für die Herstellung des Bezugs im Text verwendet wird und nicht mit der Originaldatei übereinstimmen muss.
\subsection{Verwendete Programmversionen}
In der nachstehenden Informationsbox befindet sich eine Liste mit den verwendetet Programmversionen, die zum Zeitpunkt der Anfertigung dieses Dokuments verwendet wurden. Abweichungen von den oben genannten Programmversionen können dazu führen, dass sich die in diesem Dokument beschriebenen Beispiele (Grafiken, Quellcode-Auszüge, usw.) in ihrem Vorgehen und Inhalt unterscheiden. Für die schnelle Einrichtung wird daher empfohlen nicht von den genannten Programmversionen abzuweichen. Hierunter fallen vor allem die mit einem Stern (\star) gekennzeichneten Programmversionen.
\begin{info}{010}
	Liste der verwendeten Programmversionen zum Zeitpunkt der Anfertigung des vorliegenden Dokuments:
	\begin{itemize}
		\item \href{http://www.eclipse.org/downloads/download.php?file=/technology/epp/downloads/release/mars/1/eclipse-jee-mars-1-win32-x86_64.zip}{Eclipse 4.5.1}% (Plugin: BPMN2 Modeler 1.2.1)
		%\item \href{http://download.oracle.com/otn-pub/java/jdk/8u66-b17/jdk-8u66-windows-x64.exe}{JavaJDK 1.8.66}
		\item \href{http://sourceforge.net/projects/jbpm/files/jBPM\%206/jbpm-6.3.0.Final/}{jBPM 6.3.0}\star% (Plugin: Drools and jBPM 6.3.0)
		%\item \href{http://mirror.23media.de/apache/maven/maven-3/3.3.3/binaries/apache-maven-3.3.3-bin.zip}{Maven 3.3.3}
		\item \href{http://get.enterprisedb.com/postgresql/postgresql-9.4.5-1-windows-x64.exe}{PostgreSQL 9.4.5}% (JDBC-Driver: 9.4-1205)
		\item \href{http://download.jboss.org/wildfly/9.0.2.Final/wildfly-9.0.2.Final.zip}{Wildfly 9.0.2}\star% (Plugin: JBoss Tools 4.3.0)
	\end{itemize}
\end{info}
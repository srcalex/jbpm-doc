\section{Vorbemerkungen}
Die Nachfolgenden Erläuterungen beschreiben die Installation und Konfiguration einer Eclipse-Arbeitsumgebung für jBPM im Zusammenspiel mit einem Wildfly-Applicationsever und einer PostgreSQL-Datenbank.
\subsection{Verwendete Syntax und Semantik}
Für eine bessere Verwendung dieses Dokumentes -- und eventuellen Folgedokumenten -- wurde auf eine konsistente Nutzung typographischer Stilmittel geachtet. Die folgende Auflistung stellt eine Übersicht aller in diesem Dokument betroffenen Element dar.

\begin{itemize}
	\item Downloads werden als blaue Wolke mit einem weißen, nach unten zeigenden Pfeil dargestellt: \download{arg1}. Im Allgemeinen verbirgt sich dahinter ein Direktdownload bzw. der Link zu einer Download-Seite.
	\item Quellcodebeispiele verfügen über eine gemeinsame Basis. Das Ursprungsformat des Beispiels wird durch eine differenzierte Farbwahl des linken Randes kenntlich gemacht. Im folgende werden die Ursprungsformate kurz genannt und die dazugehörige Randfarbe rechts daneben, in einem kleinen Quadrat dargestellt:
\end{itemize}
\subsection{Verwendete Programmversionen}
Zur Anfertigung dieses Dokumentes wurden folgende Programmversionen verwendet:
\begin{description}
	\item[Eclipse] (4.5.1 \download{http://www.eclipse.org/downloads/download.php?file=/technology/epp/downloads/release/mars/1/eclipse-jee-mars-1-win32-x86_64.zip}; Plugins: BPMN2 Modeler 1.2.1, Drools and jBPM 6.3.0)
	\item[JavaJDK] (1.8.66 \download{http://download.oracle.com/otn-pub/java/jdk/8u66-b17/jdk-8u66-windows-x64.exe}) 
	\item[jBPM] (6.3.0 \download{http://sourceforge.net/projects/jbpm/files/jBPM\%206/jbpm-6.3.0.Final/})
	\item[Maven] (3.3.3 \download{http://mirror.23media.de/apache/maven/maven-3/3.3.3/binaries/apache-maven-3.3.3-bin.zip})
	\item[PostgreSQL] (9.4.5 \download{http://get.enterprisedb.com/postgresql/postgresql-9.4.5-1-windows-x64.exe}; JDBC-Driver: 9.4-1205)
	\item[Wildfly] (9.0.2 \download{http://download.jboss.org/wildfly/9.0.2.Final/wildfly-9.0.2.Final.zip})
\end{description}
Bei einer Abweichung -- besonders bei älteren Versionen -- der oben genannten Programmversionen kann nicht für eine erfolgreiche Einrichtung garantiert werden.

\begin{info}{11113}
	Lorem ipsum dolor sit amet, consetetur sadipscing elitr, sed diam nonumy eirmod tempor invidunt ut labore et dolore magna aliquyam erat, sed diam voluptua. At vero eos et accusam et justo duo dolores et ea rebum. Stet clita kasd gubergren, no sea takimata sanctus est Lorem ipsum dolor sit amet. Lorem ipsum dolor sit amet, consetetur sadipscing elitr, sed diam nonumy eirmod tempor invidunt ut labore et dolore magna aliquyam erat, sed diam voluptua. At vero eos et accusam et justo duo dolores et ea rebum. Stet clita kasd gubergren, no sea takimata sanctus est Lorem ipsum dolor sit amet.
\end{info}

Und nun der Text unter der Box. Werden die Zeilen noch richtig umgebrochen oder steht wieder alles auf einer Linie.

\xmllisting{modules.xml}
\javalisting{HelloWorld.java}
\section{Vorbemerkungen}
Die Nachfolgenden Erläuterungen beschreiben die Installation und Konfiguration einer Eclipse-Arbeitsumgebung für jBPM im Zusammenspiel mit einem Wildfly-Applicationsever und einer PostgreSQL-Datenbank. Zur Anfertigung dieses Dokumentes wurden folgende Programmversionen verwendet:
\begin{description}
	\item[Eclipse] (4.5.1 \download{http://www.eclipse.org/downloads/download.php?file=/technology/epp/downloads/release/mars/1/eclipse-jee-mars-1-win32-x86_64.zip}; Plugins: BPMN2 Modeler 1.2.1, Drools and jBPM 6.3.0)
	\item[JavaJDK] (1.8.66 \download{http://download.oracle.com/otn-pub/java/jdk/8u66-b17/jdk-8u66-windows-x64.exe}) 
	\item[jBPM] (6.3.0 \download{http://sourceforge.net/projects/jbpm/files/jBPM\%206/jbpm-6.3.0.Final/})
	\item[Maven] (3.3.3 \download{http://mirror.23media.de/apache/maven/maven-3/3.3.3/binaries/apache-maven-3.3.3-bin.zip})
	\item[PostgreSQL] (9.4.5 \download{http://get.enterprisedb.com/postgresql/postgresql-9.4.5-1-windows-x64.exe}; JDBC-Driver: 9.4-1205)
	\item[Wildfly] (9.0.2 \download{http://download.jboss.org/wildfly/9.0.2.Final/wildfly-9.0.2.Final.zip})
\end{description}
Bei einer Abweichung -- besonders bei älteren Versionen -- der oben genannten Software kann nicht für eine erfolgreiche Einrichtung garantiert werden.

Das ist ein kleiner Absatz und hier steht etwas Text, um den Absatz mindestens ein, zwei Zeilen länge zu spendieren.

Mal sehen, ob das github jetzt den richtigen Benutzernamen und die richtige E-Mail-Adresse erkennt.

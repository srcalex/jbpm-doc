\section{Vorbemerkungen}
Die Nachfolgenden Erläuterungen beschreiben die Installation und Konfiguration einer Eclipse-Arbeitsumgebung für jBPM im Zusammenspiel mit einem Wildfly Application Sever und einer PostgreSQL-Datenbank.

Für eine bessere Verwendung dieses Dokumentes -- und eventuellen Folgedokumenten -- wurde auf eine konsistente Nutzung typographischer Stilmittel geachtet, welche im folgenden kurz erläutert werden sollen.

Zusätzliche Informationen -- welche nicht zwingend für die Bearbeitung des aktuellen Kapitels, Abschnitts bzw. Unterabschnitts notwendig sind -- werden jeweils am Ende in einer separaten Informationsbox eingefügt. Der Inhalt kann sich dabei je nach Bedarf unterscheiden und wird über eine Variation verschiedener Symbole in der oberen, rechten Ecke der Informationsbox beschrieben. Besonders bei längeren Informationsboxen soll dies dem Leser helfen, den darin befindlichen Inhalt schnell zu überblicken und die freie Entscheidung über ein möglicherweise gewünschtes Lesen beschleunigen. Das Aussehen und eine Übersicht aller verwendeten Symbole sowie deren Bedeutung zeigt die unten zu sehende Informationsbox.
\begin{info}{111}
	Erläuterungen zur Bedeutung der Symbole:\\[1em]
	\begin{tabular}{m{2em}m{0.88\textwidth}}
	\bookmarkicon	& deutet an, dass die Informationsbox Verweise zu externen Quellen mit zusätzlichen Informationen zu diesem Thema beinhaltet. \\[0.5em] 
	\downloadicon	& deutet an, dass die Informationsbox einen oder mehrere Verweise zu externen Download-Quellen beinhaltet. \\[0.5em]
	\fileicon	& deutet an, dass die Informationsbox zusätzliche Erläuterungen zu dem im aktuellen Kapitel, Abschnitt oder Unterabschnitt beinhaltet.
	\end{tabular}
\end{info}
Quellcode-Beispiele werden unabhängig von der verwendete Programmier- bzw. Script-Sprache einheitlich dargestellt. Beispielhaft soll dies im unten zu sehenden Listing \ref{lst:listing-example} dargestellt werden. 
\javalisting{HelloWorld.java}{listing-example}
Der Bezug im Text wird über die Zeilennummer vorgenommen, welche sich am linken Rand des Listings befinden und stets mit 1 beginnt. Bei Quellcode-Auszügen kann sich die Zeilennummer vom Original unterscheiden und der Inhalt bzw. die Einfügestelle muss entsprechend gesucht werden.
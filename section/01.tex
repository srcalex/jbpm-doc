\section{Einrichtung der Eclipse-Arbeitsumgebung}
Im Folgenden wird davon ausgegangen, dass die Eclipse IDE (for Java EE Developers) in der oben genannten Programmversion installiert ist.

\subsection{Installation empfohlener Eclipse Plugins}
Die Installation zusätzliche Plugins in Eclipse erfolgt über den Menüpunkt \textsc{Help / Install New Software\ldots}. In dem sich nun öffnenden Fenster kann eine \textsc{Software Site} des zu installierenden Plugins angegeben werden. Diese wird in das Feld \textsc{Work with:} eingetragen und nach einem klick auf \textsc{Add\ldots} durch Angabe eines Namens dauerhaft zu den verfügbaren \textsc{Software Sites} der aktuellen Eclipse-Installation hinzugefügt. Gleichzeitig werden alle unter dieser \textsc{Software Site} verfügbaren Plugins im unteren Bereich des Fensters aufgelistet. Durch die Auswahl der benötigten Plugins und einem Klick auf \textsc{Next} können die ausgewählten Plugins installiert werden. Die nun erscheinende Lizenzvereinbarung ist mit einem Klick auf \textsc{Ok} zu bestätigen.

\subsubsection{BPMN2 Modeler}
Die aktuelle Version des BPMN2 Modeler ist unter \url{http://download.eclipse.org/bpmn2-modeler/updates/mars/1.2.1/}. Für die Nutzung des BPMN2 Modelers in Verbindung mit jBPM ist folgendes Plugin zu installieren:
\begin{itemize}\renewcommand{\labelitemi}{\itemizecheck}
	\item BPMN2 Modeler - jBPM Runtime Extention Features
\end{itemize}
Nach der Installation ist es möglich {\ttfamily.bpmn} und {\ttfamily.bpmn2} Dateien mit dem BPMN2 Modele Editor bzw. BPMN2 Diagram Editor zu öffnen.

\subsubsection{JBoss jBPM}
Die für eine Installation von JBoss jBPM benötigte \textsc{Software Site} lautet: \url{http://downloads.jboss.org/jbpm/release/6.3.0.Final/updatesite/}. Für die weiteren Erläuterungen ist lediglich die Installation eines folgenden Plugins notwendig:
\begin{itemize}\renewcommand{\labelitemi}{\itemizecheck}
	\item JBoss jBPM Core
\end{itemize}
Im Anschluss kann unter dem Menüpunkt \textsc{Window / Preferences} im sich nun öffnenden Fester, der Eintrag \textsc{jBPM / Installed jBPM Runtime} ausgewählt und der Speicherort für die jBPM-Runtime angegeben werden. Diese Schritt ist optional (siehe Informationsbox).

\subsection{JBoss Tools}
JBoss Tools stellt eine Sammlung verschiedene Plugins zur Verfügung, beispielsweise um in Eclipse einen installierten Wildfly Application Server zu steuern. Die \textsc{Software Site} lautet: \url{http://download.jboss.org/jbosstools/mars/stable/updates/}. Das zu installierende Plugin kann unter dem Namen:
\begin{itemize}\renewcommand{\labelitemi}{\itemizecheck}
	\item JBoss AS, WildFly \& EAP Server Tools
\end{itemize}
über die Suche gefunden werden. Nach der Installation ist es unter anderem möglich den WildFly Application Server zu starten, zu stoppen oder Projekte direkt aus Eclipse heraus zu de- und unployen. Hierfür sind dem Plugin zunächst die bereitstehenden WildFly Application Server bekannt zumachen. Dies erfolgt über den Menüpunkt \textsc{Window / Preferences} unter dem Eintrag \textsc{JBoss Runtime Detection}. Unterhalb der hier angegebenen Verzeichnisse erfolgt eine automatische Suche nach bereits installierte WildFly Application Servern. Alle gefundenen WildFly Application Server stehen anschließend über die Server-View in Eclipse zur Verfügung. Zusätzliche Serverkonfigurationen können über \textsc{Window / Preferences} im Bereich Server hinzugefügt werden.
\begin{info}{100}
	Das setzen der \textsc{jBPM Runtime} ist nur notwendig, wenn jBPM-Projekte ohne Maven erzeugt werden. Nach dem Setzen der \textsc{jBPM Runtime} wird diese automatisch während der Erstellung eines neuen jBPM-Projekts hinzugefügt.
\end{info}